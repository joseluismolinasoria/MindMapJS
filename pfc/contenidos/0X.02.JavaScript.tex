\subsection{JavaScript}

Cuando en 1996, el navegador Netscape introdujo su primer interprete de
JavaScript\footnote{JavaScript fue un nombre por conveniencia legal. Originalmente se llamaba
  LiveScript} nadie podía intuir la importancia que adquiriría años después. 

Internet aun estaba en pañales, navegar era lento\footnote{La velocidad máxima de los modems de
  usuario era 28.8Kbps} y los ordenadores personales poco potentes. En el mejor de los casos, el
usuario tenía que esperar durante largo tiempo para poder interactuar con la web solicitada.  Las
páginas comenzaban a ser más complejas, y la navegación más lenta, de ello surguió la necesidad de
un lenguaje de programación que se ejecutará en el navegador del cliente. De esta forma, si el
usuario introducía un valor incorrecto, en un formulario, no tendría que esperar a la respuesta del
servidor, el mismo cliente podría dar una respuesta más rápida, indicando los errores existentes.

Netscape Navigator 3.0 incorporó la primera versión del lenguaje, como ya se había comentado, y al
mismo tiempo, o al poco, Microsofot lanzo JScript en su Internet Explorer 3. JScript no erá más que 
una copia de JavaScript al que le cambiaron el nombre para evitar problemas legales. De esta
forma comienzan las divergencias entre las distintas versiones de JavaScript, en esencia todas
parten del mismo lenguaje y estandar, pero cada una aportaba sus mejoras provocando diferencias
entre ellas. 

La guerra entre las distintas versiones estaba servida. Todos deseaban que su versión fuera la
aceptada por la comunidad y se popularizará. Bien intentando estandarizar su versión, o buscando
que se evitará la guerra de tecnologías, Netscape decidió dar el paso, y en 1997 puso a disposición
de ECMA\footnote{European Computer Manufacturers Association. Web oficial
  http://www.ecma-international.org/} la especificación de JavaScript1.1. ECMA creo el comité TC39
del cual surgío el primer estándar que se denominó ECMA-262\footnote{Se puede encontrar la versión 5.1 en
  http://www.ecma-international.org/publications/files/ECMA-ST/ECMA-262.pdf}, o más popularmente, 
ECMAScript. 

Durante mucho tiempo el estándar ECMAScript no fue el aceptado por todos los navegadores, ni que
decir tiene que el más reacio al cambio fue el Internet Explorer de Microsoft. Es ahora, donde
Microsoft a dado su brazo a torcer y poco a poco tiende al estándar ECMAScript facilitando al los
desarrolladores la tarea.
