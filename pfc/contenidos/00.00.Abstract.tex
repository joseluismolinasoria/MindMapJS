\newpage\mbox{}\thispagestyle{empty}

\chapter{Abstract}

El editor de mapas mentales on-line es un sistema web desarrollado única y exclusivamente en HTML5
para diseñar y elaborar mapas mentales con formato FreeMind. Formato el cual, es considerado un
estándar en el mundo de los mapas mentales.

La idea que subyace en la realización de este editor de mapas mentales, es probar las nuevas
tecnologías existentes alrededor de HTML5 y comprobar el estado actual de dicha
tecnología tras el interés y la relevancia que está adquiriendo en estos últimos años. 

La metodología Ágil, utilizada para llevar a cabo el proyecto, se adapta muy bien al desarrollo web. Permitiendo un feedback constante desde la versión inicial hasta la actual. Conjuntamente con la metodología Ágil se utilizado otras metodologías y paradigmas de programación como el BBD\footnote{Como sistema de pruebas y verificación del código fuente}, patrones de diseño, etc ... También se ha hecho uso de tecnologías ya existentes como soporte al desarrollo, entre ellas destacar KineticJS, Mocha, GruntJS, NodeJS, JSHint, JSDoc y GitHub. Todas estas tecnologías han propiciado una experiencia de desarrollo satisfactoria. 

En resumen, se ha podido constatar un gran avance en HTML5 con respecto a la versión anterior. A pesar de ello, sigue existiendo, aunque en mucho menor grado, una gran dependencia con el navegador. 
